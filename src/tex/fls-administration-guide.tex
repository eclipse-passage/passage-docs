% Copyright (c) 2021 ArSysOp
%
% This program and the accompanying materials are made available under the
% terms of the Eclipse Public License 2.0 which is available at
% https://www.eclipse.org/legal/epl-2.0/.
%
% SPDX-License-Identifier: EPL-2.0

\documentclass[12pt]{report}

\usepackage{color}
\usepackage[usenames,dvipsnames,svgnames,table]{xcolor}

\usepackage{graphicx}
\graphicspath{{../../i/common/}}

\usepackage{hyperref}
\hypersetup{colorlinks=true,
linkcolor=black,
urlcolor=gray}

\usepackage{fancyhdr}
\pagestyle{fancy}
\fancyhf{}
\lhead{\includegraphics[width=13pt]{glyph}}
\rhead{\leftmark}
\lfoot{\thepage}
\rfoot{Eclipse Passage}
\renewcommand{\chaptermark}[1]{\markboth{#1}{}}
\renewcommand{\headrulewidth}{0.5pt}
\renewcommand{\footrulewidth}{0.5pt}

\fancypagestyle{plain}{
    \fancyhf{}
    \lhead{\includegraphics[width=13pt]{glyph}}
    \rhead{\leftmark}
    \lfoot{\thepage}
    \rfoot{Eclipse Passage}
    \renewcommand{\headrulewidth}{0.5pt}
    \renewcommand{\footrulewidth}{0.5pt}
}

\setlength{\headheight}{16pt}

\title{Eclipse Passage Floating License Server}
\author{ArSysOp}
\date{07 Jule 2021}

\begin{document}

\begin{titlepage}
    \begin{center}
        \vspace*{1cm}

        \Huge \textbf{Eclipse Passage}
        
        \Huge \textbf{Floating License Server}
        
        \Huge \textbf{Administration Guide}

        \vspace{0.5cm}

        \Large \today

        \vfill

        \includegraphics[width=5cm]{passage}

        \vfill

        \Large ArSysOp
    \end{center}
\end{titlepage}

\tableofcontents
\markboth{Table of Contents}{}

\addcontentsline{toc}{chapter}{Overview}
\chapter*{Overview}
Eclipse Passage is a set of tools and libraries providing rich and easily adaptable capabilities 
to declare and control licensing constraints.

\addcontentsline{toc}{section}{Contribution}
\section*{Contribution}
If you found a mistake in the text or just want to improve this user guide, feel free to contribute on \href{https://github.com/eclipse-passage/passage-docs}{Github} repository.

Also we want to remind that Eclipse Passage is an open-source project, so you can always contribute to the project \href{https://github.com/eclipse-passage/passage}{itself}.

\addcontentsline{toc}{section}{License}
\section*{License}
These materials are made available under the terms of the \href{https://www.eclipse.org/legal/epl-2.0/}{Eclipse Public License 2.0}.

\addcontentsline{toc}{chapter}{How floating licensing works}
\chapter*{How floating licensing works}

As opposed to personal licensing, where a user of a product-under-licensing 
gets their own license and hosts it close to the product for single use,
floating licensing delegates actual license packs hosting and executing to a dedicated agent - \textbf{Floating License Server}. 

This allows number of users to exploit their installations of the product simultaneously in the scope of a single floating license pack. 

\addcontentsline{toc}{section}{Floating License Server}
\section*{Floating License Server}

The Server, that is installed somewhere in the reach, 
\begin{itemize}
    \item hosts all floating license packs and
    \item responds to \textit{give me permission to use this feature} requests from the product's users via http.
\end{itemize}

\addcontentsline{toc}{section}{Communication with the Server}
\section*{Communication with the Server}

The product 
\begin{itemize}
    \item asks the Server for a grant on a particular feature and, 
    \item if succeed, proceeds with the feature and then, 
    \item when it's over, releases the grant.
\end{itemize} 

The Server can lease several grants for the same feature at the same time, as hosted floating licence packs permit.

\addcontentsline{toc}{section}{Full floating licensing workflow}
\section*{Full floating licensing workflow}

There are four \textit{actors} in the floating licensing workflow:

\begin{itemize}
	\item \textit{product development and management}, where
		\begin{itemize}
		    \item management \textit{defines products}, \textit{feature sets} and \textit{product release plan} and
		    \item development
		    	\begin{itemize}
		      		\item \textit{declares licensing requirements} in the product codebase,
		      		\item  \textit{covers feature usages with licensing protection},
		      		\item  \textit{configures the product to use Floating License Server};
		      	\end{itemize}
      	\end{itemize}
	\item \textit{operator}, who issues a~\hyperref[ch:flp]{floating license pack} and delivers it,  
	\item \textit{agent}, who~\hyperref[ch:install-fls]{installs} and~\hyperref[ch:run-fls]{runs} a Floating License Server,
	\item \textit{client} as a representative for a set of phisical users of the product \textit{users}, which are enlisted in a \textit{floating license pack}.

\end{itemize} 

\addcontentsline{toc}{chapter}{What is in a Floating License Pack}
\chapter*{What is in a Floating License Pack} \label{ch:flp}

\addcontentsline{toc}{section}{Pack}
\section*{Pack} \label{sec:flp-pack}

When operator issues a floating license pack, it resides in a folder named after the \textit{pack identifier} like 
\textit{d2b83215-b65d-4031-a8c8-a10421d56260}. Name of each file in the folder starts with this identifier as well. 
The folder contains all necessary data for all parties involved in floating licensing. 

\addcontentsline{toc}{section}{Files}
\section*{Files} \label{sec:flp-files}
There are in the folder
\begin{itemize}
	\item \textbf{license files}, which contain list of all users eligible for the license and any feature can be used by it; they should be ~\hyperref[sec:fls-commands-upload]{passed under control} of \textbf{Floating License Server} 
	\item optionally, \textbf{floating access files}, one per user, which keep connection and authentication credentials of the Server; there files are to be passed to each user mentioned in floating the license pack.
\end{itemize}

Table \ref{tabular:sample-pack-content} shows a typical file structure of a floating license pack, delivered to a client.
\begin{table}[ht]
\caption{Sample floating license pack content}
\label{tabular:sample-pack-content}
\begin{center}
	\begin{tabular}{p{0.2\linewidth}p{0.2\linewidth}p{0.5\linewidth}}
		\textbf{File} & \textbf{Role} & \textbf{Action} \\  \hline
		\textit{pack-id}.flicen & Encrypted license & Pass to agent to be deployed under Floating License Server. \\  \hline
		\textit{product-id}\_\textit{product-version}.pub & Product's public key & Pass to agent to be deployed under Floating License Server, together with the license file. \\ \hline  
		\textit{pack-id}\_\textit{user-id}.flaen & Encrypted floating access file & Pass to the mentioned user to be stored close to their installation of the product. \\  \hline
	\end{tabular}
\end{center}
\end{table}

\addcontentsline{toc}{section}{Content}
\section*{Content} \label{sec:flp-content}

\addcontentsline{toc}{subsection}{Floating license file}
\subsection*{Floating license file} \label{sec:flp-content-lic}
Floating license File \textit{*.flicen} accumulates
\begin{itemize}
  \item \textbf{Floating License Server authentication} information, as each floating license is to be released for a particular residence. A license is not intended to be functional in case it is deployed on a server, that cannot be authenticated with these credentials;
  \item enlistment of all \textbf{users allowed to exploit the product} - their identifiers, authentication conditions, other info;
  \item list of product features that can be used in the scope of this license pack, associated with the amount of available grants (simultaneous usages).
\end{itemize}

A product, running on a \textit{user}'s installation, \textit{acquires a grant} for a \textit{feature}, 
which is valid for short period (\textit{vivid} for 60 minutes by default). 
Since a grant is acquired, it's not accessible until it is released. 

After the \textit{feature}'s work is over, the product \textit{releases the grant} and thus it becomes available for acquisition again.

\addcontentsline{toc}{subsubsection}{Feature grant capacity}
\subsubsection*{Feature grant capacity} \label{sec:flp-content-capacily}
Several product installations (users) can acquire different grants for the same feature for overlapping time slots. 
Feature grant's \textit{capacity} field defines this amount.
If there are no more available grants, the acquisition fails, and the product blocks the feature utilization. 

\addcontentsline{toc}{subsubsection}{Feature grant vivid}
\subsubsection*{Feature grant vivid} \label{sec:flp-content-vivid}
A feature grant, been leased for a user, should not last long. 
Thus, a feature grant information contain instruction on how \textit{vivid} is the grant. 
Measured in minutes, it defines how long will the grant be valid after leased (limited by the whole license pack validity period).
After this relatively short vivid period the grant is not active, but still acquired until is explicitly released.

\addcontentsline{toc}{subsubsection}{Feature grant validity period}
\subsubsection*{Feature grant validity period} \label{sec:flp-content-valid}
A feature grant can define its own validity period. 
Even if it is declared to be wider than the owning license pack validity period, at runtime the former is still limited by the latter.
This feature is used by licensing party to narrow validity period for a particular feature in the pack.

\addcontentsline{toc}{subsection}{Floating access file}
\subsection*{Floating access file} \label{sec:flp-content-access}
A product installation, been located on a user's workspation, needs to be told how to access a Floating License Server.

Floating access file delivers this information. It contains:
\begin{itemize}
  \item user identifier,
  \item the Server connection url components: ip address and port,
  \item the Server authentication condition.
\end{itemize} 

Each user enlisted in the license file should posess such an access file and host it where the product is configured to look for it.
Usually it is the product's folder under default \textit{.passage} data directory: \textit{user-home}/.passage/\textit{product-id}/\textit{product-version} 

\addcontentsline{toc}{section}{Destination}
\section*{Destination} \label{sec:flp-destination}

Floating license files are to be ~\hyperref[sec:fls-commands-upload]{put under control} of the Floating License Server for which this floating license pack has been issued.

\addcontentsline{toc}{chapter}{Install Floating License Server}
\chapter*{Install Floating License Server} \label{ch:install-fls}

Floating License Server is distributed in OS-specific archives, which does not need to be installed, just unpacked. 

Nevertheless, server executable should be granted with all the necessary permissions
\begin{itemize}
  \item to its working directory,
  \item to the directory where the Server hosts license packs.
\end{itemize} 

The Server is implemented over \textbf{Java 11}, 
so make sure that \textit {java -version} reports proper version and this runtime is available for the server executable.

\addcontentsline{toc}{chapter}{Run Floating License Server}
\chapter*{Run Floating License Server} \label{ch:run-fls}

Start the Server executable located in the root of the unpacked content. 
It triggers \textit{osgi} console and there immediately starts the Server. The console is used for the Server administration.

\addcontentsline{toc}{section}{Command line parameters}
\section*{Command line parameters} \label{ch:run-fls-cli}

Server runtime can be configured with the command line parameters for the Server executable, enlisted in Table \ref{tabular:server-params}:

\begin{table}[ht]
\caption{Floating License Server command line parameters}
\label{tabular:server-params}
\begin{center}
	\begin{tabular}{p{0.2\linewidth}p{0.26\linewidth}p{0.5\linewidth}}
		\textbf{Parameter} & \textbf{Default} & \textbf{Usage} \\  \hline
		\textit{-server.port} & 8090 & \textit{-server.port=8089} will start the Server which will use the specified port for http communications.\\  \hline
		\textit{-server.storage} & \textit{user-home}/.passage & Path to directory where the Server is expected to keep floating licenses packs. \\  \hline
	\end{tabular}
\end{center}
\end{table}

\addcontentsline{toc}{section}{Commands}
\section*{Commands} \label{ch:run-fls-commands}

\addcontentsline{toc}{subsection}{Lifecycle commands}
\subsection*{Lifecycle commands} \label{ch:run-fls-commands-lc}
 
The Server can be stopped and restarted from the console. 

\begin{table}[h!t]
\caption{Server Administration Commands | Lifecycle}
\label{tabular:server-commands-lc}
\begin{center}
	\begin{tabular}{p{0.3\linewidth}p{0.65\linewidth}}
		\textbf{Command} & \textbf{Output} \\  \hline
		\textgreater \textit{ fls:stop} & Stops the Server\\  \hline
		\textgreater \textit{ fls:start} & Starts the Serve.\\  \hline
		\textgreater \textit{ fls:restart} & Restarts the Server\\  \hline
		\textgreater \textit{ fls:state} & Reports either the Server is currently started or stopped\\ \hline
		\textgreater \textit{ exit} & Terminates the Server and exit the administration console\\ \hline
	\end{tabular}
\end{center}
\end{table}

\addcontentsline{toc}{subsection}{Self licensing commands}
\subsection*{Self licensing commands} \label{ch:run-fls-commands-lc}

Floating License Server itself is a product licensed with Passage. 
That means you should put proper personal license for the Server under 
\textit{user-home}/.passage/\textit{server-product-id}/\textit{server-product-version} folder.

\begin{table}[ht]
\caption{Server Administration Commands | Self Licensing}
\label{tabular:server-commands-lic}
\begin{center}
	\begin{tabular}{p{0.3\linewidth}p{0.65\linewidth}}
		\textbf{Command} & \textbf{Output} \\  \hline
		\textgreater \textit{ self:licstatus} & Reports current licensing status of the Floating License Server product.\\  \hline
		\textgreater \textit{ self:licrequest} & Reports complete platform assessment, which is necessary to request a license for the Server from a licensing operator. \\  \hline
	\end{tabular}
\end{center}
\end{table}

\addcontentsline{toc}{subsection}{Manage Floating License Packs}
\subsection*{Manage Floating License Packs} \label{sec:run-fls-commands-lp}

\addcontentsline{toc}{subsubsection}{Upload Floating License Pack }
\subsubsection*{Upload Floating License Pack} \label{sec:run-fls-commands-lp-u}

Use \textit{fls:upload} command to put a floating license pack under the Server control. 

\textgreater \textit{ fls:upload path/to/a/fl-pack-folder/like/0803cf43-b9ce-4051-bcb8-15ccf00f036b/}


The Server reads the pack and places \textit{*.flicen} and \textit{*.pub} files under the directory, 
~\hyperref[ch:run-fls-cli]{condifured} by \textit{storage} parameter. 

\end{document}
