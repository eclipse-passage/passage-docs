% Copyright (c) 2021 ArSysOp and others
%
% This program and the accompanying materials are made available under the
% terms of the Eclipse Public License 2.0 which is available at
% https://www.eclipse.org/legal/epl-2.0/.
%
% SPDX-License-Identifier: EPL-2.0
%
% Contributors:
%     Nikifor Fedorov (ArSysOp)

\documentclass[12pt]{report}

\usepackage{color}
\usepackage[usenames,dvipsnames,svgnames,table]{xcolor}

\usepackage{graphicx}
\graphicspath{{./images/}}

\usepackage{hyperref}
\hypersetup{colorlinks=true,
linkcolor=black,
urlcolor=gray}

\usepackage{fancyhdr}
\pagestyle{fancy}
\fancyhf{}
\lhead{\includegraphics[width=11pt]{passage}}
\rhead{Eclipse Passage}
\lfoot{\thepage}
\rfoot{Arsysop}
\renewcommand{\headrulewidth}{0.5pt}
\renewcommand{\footrulewidth}{0.5pt}

\fancypagestyle{plain}{
    \fancyhf{}
    \lhead{\includegraphics[width=11pt]{passage}}
    \rhead{Eclipse Passage}
    \lfoot{\thepage}
    \rfoot{Arsysop}
    \renewcommand{\headrulewidth}{0.5pt}
    \renewcommand{\footrulewidth}{0.5pt}
}

\setlength{\headheight}{16pt}

\title{Eclipse Passage User Guide}
\author{ArSysOp}
\date{26 June 2021}

\begin{document}

\begin{titlepage}
    \begin{center}
        \vspace*{1cm}

        \Huge
        \textbf{Eclipse Passage User Guide}

        \vspace{0.5cm}

        \Large
        \today

        \vfill

        \includegraphics[width=5cm]{passage}

        \vfill

        \Large
        ArSysOp
    \end{center}
\end{titlepage}

\tableofcontents

\addcontentsline{toc}{chapter}{Introduction}
\chapter*{Introduction}

\addcontentsline{toc}{section}{Overview}
\section*{Overview}

\textbf{Eclipse Passage} is a set of tools providing rich and easily adaptable capabilities to declare and control licensing constraints.
Being an open-source project under Eclipse Foundation, it is entirely developed with Java 11 by the \textit{Arsysop} company.

This user guide contains a glossary, which gives you base understanding of terms used in Eclipse Passage and two sets of instructions: for developer, who
wants to license some features or entire application with Passage, and for operator, who wants to manage licenses and metadata with the Eclipse Passage Operator.

Happy Reading!

\addcontentsline{toc}{section}{Contribution}
\section*{Contribution}

If you found a mistake in the text or just want to improve this user guide, feel free to contribute on \href{https://github.com/eclipse-passage/passage-docs}{Github} repository.

Also we want to remind that Eclipse Passage is an open-source project, so you can always contribute to the project \href{https://github.com/eclipse-passage/passage}{itself}.

\addcontentsline{toc}{section}{License}
\section*{License}

These materials are made available under the terms of the \href{https://www.eclipse.org/legal/epl-2.0/}{Eclipse Public License 2.0}.

\addcontentsline{toc}{chapter}{Components}
\chapter*{Components}

Eclipse Passage consists of three parts:

\addcontentsline{toc}{section}{Licensing Components}
\section*{Licensing Components}

Licensing Components is a set of tools you have to include in your application to be able to declare any licensing logic (For example, what to do if a correct license was not acquired by the licensing framework).
It allows you to restrict unauthorized using of a single feature, bundle or even the whole application.

\addcontentsline{toc}{section}{Operator}
\section*{Operator}

Operator client - is a separate application that gives you a manual control under all the licenses your product can have (both personal and floating). Also, it is required
in access cycle in order to define the product or/and features metadata and to generate keypair for license file encoding.

\addcontentsline{toc}{section}{Floating License Server}
\section*{Floating License Server}

Eclipse Passage FLS is a component implementing the Floating License Server concept. Works roughly in a way described below.

The licensee acquires (in any way) a finite number of licenses and these licenses are stored on the License Server. When an authorized user wants to use the application,
they request a license from the server, and if the license pool is not empty (in other words, there are still licenses available), the server gives user an access to the
application. When user finishes his work with the application, the license is released and can be obtained by the other authorized user.

\addcontentsline{toc}{chapter}{Developer's guide}
\chapter*{Developer's guide}

This section is under construction for now.

\addcontentsline{toc}{chapter}{Eclipse Operator User Guide}
\chapter*{Eclipse Operator User Guide}

\addcontentsline{toc}{section}{User Interface Overview}
\section*{User Interface Overview}



\addcontentsline{toc}{chapter}{Common Problems}
\chapter*{Common Problems}

This section is under construction for now.

\addcontentsline{toc}{chapter}{Glossary}
\chapter*{Glossary}

This section is under construction for now.

\end{document}